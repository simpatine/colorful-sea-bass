\section{Methodology}
\label{sec:methods}

In this work we tried to evaluate the effectiveness of random subsampling of genes in improving predictions of mortality.
We always subsampled selecting a subset of genes in the whole dataset.

The idea is to help our estimator to not get lost in the $\sim 10^6$ genes, but allowing it to work with only a smaller number of genes at a time. Hopefully in this way our model will give better results.

Once we subsampled the dataset, we used the \texttt{XGBoost}\cite{xgboost} library to construct a predictor in a fast and easy manner. 

\paragraph{Uniform subsampling.}
The first and simplest thing we tried was to randomly and uniformly subsample the genes on the entire genome.
We selected a given fraction of the genes, keeping or discarding each gene with fixed and uniform probability.

\paragraph{Uniform subsampling on chromosomes.}
A second possibility is to uniformly sample a fixed fraction of genes on each chromosome.

\paragraph{Subsampling using annotations.}
Some genes have been linked to specific organs, tissues or functions. This information can be used to select and sample genes
