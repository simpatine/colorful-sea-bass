\documentclass[aspectratio=169]{beamer}

\usepackage{amsmath,amssymb,amsfonts,amsthm}
\usepackage[T1]{fontenc}
\usepackage[utf8]{inputenc}
\usepackage[english]{babel}
\usepackage{hyperref}
\usepackage{lmodern}
\usepackage{comment}
\usepackage{xcolor}
\usepackage{graphicx,subcaption}
\usepackage[linesnumbered,ruled,vlined,noend,algo2e]{algorithm2e}
\usepackage{tcolorbox}

\renewcommand{\epsilon}{\varepsilon}
\newcommand{\R}{\mathbb{R}}
\newcommand{\N}{\mathbb{N}}
\newcommand{\B}[2]{\mathcal{B}_{#1}\left(#2\right)}
\newcommand{\todo}[1]{{\color{red}#1}}

% Use Unipd as theme, with options:
% - pageofpages: define the separation symbol of the footer page of pages (e.g.: of, di, /, default: of)
% - logo: position another logo near the Unipd logo in the title page (e.g. department logo), passing the second logo path as option 
% Use the environment lastframe to add the endframe text
\usetheme[pageofpages=of]{Unipd}
\usefonttheme[onlymath]{serif}

\title{Random subsampling techniques for sea bass mortality prediction}
\author{Giovanni Gaio, Simone Moretti}
\date{\today}
\subtitle{}

\date{\today}

% The next block of commands puts the table of contents at the beginning of each section and highlights the current section

\begin{comment}
\AtBeginSection[]
{
  \begin{frame}
    \frametitle{Table of Contents}
    \tableofcontents[currentsection]
  \end{frame}
}
\end{comment}


\begin{document}

\frame{\titlepage}

\begin{frame}
\frametitle{Overview}
\begin{itemize}
  \item Motivation: Identifying impactful SNPs in sea bass mortality
  \item Dataset: Genomic SNP data, mortality outcomes, and annotations
  \item Method: Subsampling techniques with XGBoost
  \item Results: Accuracy/F1 vs. subsampling rate
  \item Conclusion: Subsampling preserves predictive power
\end{itemize}
\end{frame}

\begin{frame}
\frametitle{SNPs and Sea Bass Mortality}
\begin{itemize}
  \item VNN is a widespread lethal disease in sea life.
  \item SNPs may increase or decrease the chance of death.
  \item Goal: Predict mortality based on SNP profiles.
\end{itemize}
\end{frame}

\begin{frame}
\frametitle{Challenges with Genomic Data}
\begin{itemize}
  \item Each fish: over 6 million SNP positions.
  \item Sample size: only 990 sea bass individuals.
  \item Traditional models overfit due to data dimensionality.
\end{itemize}
\end{frame}

\begin{frame}
\frametitle{Machine Learning Approach}
\begin{itemize}
  \item Use XGBoost classifier for mortality prediction.
  \item Need to reduce feature space: apply subsampling.
  \item Evaluate performance on subsampled datasets.
\end{itemize}
\end{frame}

\begin{frame}
\frametitle{Pipeline Overview}
%\includegraphics[width=\linewidth]{pipeline_diagram.png} % placeholder for Figure 1
\end{frame}

\begin{frame}
\frametitle{SNP Dataset Structure}
\begin{itemize}
  \item 990 rows (fish), each with 6,072,853 SNP features.
  \item SNP values: 0 (no mutation), 1 (heterozygous), 2 (homozygous alt).
  \item Each fish is paired with a mortality label.
\end{itemize}
\end{frame}

\begin{frame}
\frametitle{Annotation Metadata}
\begin{itemize}
  \item Annotations include function: Promoter, Enhancer, Open Chromatin.
  \item Tissue number (0–25) indicates location-specific relevance.
\end{itemize}
\end{frame}

\begin{frame}
\frametitle{Uniform Subsampling}
\begin{itemize}
  \item Randomly sample a fixed proportion $p$ of all SNPs.
  \item Simple but may cause imbalance across chromosomes.
\end{itemize}
\end{frame}

\begin{frame}
\frametitle{Per-Chromosome Subsampling}
\begin{itemize}
  \item Ensures balanced representation from each chromosome.
  \item Randomly sample same number of SNPs per chromosome.
\end{itemize}
\end{frame}

\begin{frame}
\frametitle{Annotation-Based Subsampling}
\begin{itemize}
  \item Filter SNPs by biological annotation.
  \item Then apply uniform subsampling to relevant regions.
\end{itemize}
\end{frame}

\begin{frame}
\frametitle{Subsampling Strategy Comparison}
\begin{itemize}
  \item Trade-offs in simplicity, biological interpretability, and balance.
  \item Aim: maximize predictive power while reducing dimensionality.
\end{itemize}
\end{frame}

\begin{frame}
\frametitle{XGBoost Training}
\begin{itemize}
  \item Training/testing split is fixed before subsampling.
  \item Trained multiple times per method/rate to average performance.
\end{itemize}
\end{frame}

\begin{frame}
\frametitle{Control of Randomness}
\begin{itemize}
  \item XGBoost random seed fixed.
  \item Subsampling is the only random step.
\end{itemize}
\end{frame}

\begin{frame}
\frametitle{Subsampling Ratios}
\begin{itemize}
  \item Subsampled with multiple $p$ values (e.g. 0.01, 0.05, 0.1, 0.2).
  \item Trained model for each combination.
\end{itemize}
\end{frame}

\begin{frame}
\frametitle{Results: Genome-wide Subsampling}
\begin{itemize}
  \item Accuracy and F1 show mostly flat trend across $p$ values.
  \item Slight performance drop at very low rates ($p < 0.01$).
\end{itemize}
\end{frame}

\begin{frame}
\frametitle{Results: Annotated Regions}
\begin{itemize}
  \item Scores consistent across function (e.g., Promoter vs Enhancer).
  \item Subsampling by tissue shows minor fluctuations.
\end{itemize}
\end{frame}

\begin{frame}
\frametitle{Interpretation of Results}
\begin{itemize}
  \item Model performance largely independent of subsampling rate.
  \item Low $p$ reduces predictive quality — not surprising.
  \item Subsampling improves speed without sacrificing accuracy.
\end{itemize}
\end{frame}

\begin{frame}
\frametitle{Conclusions}
\begin{itemize}
  \item Random SNP subsampling retains model effectiveness.
  \item No strong trend between rate and accuracy (outside extremes).
  \item Enables faster, scalable experimentation for genomic prediction.
\end{itemize}
\end{frame}

\begin{frame}
\frametitle{Thank You / Questions?}
\centering
\Huge Questions?
\end{frame}

\end{document}


\end{document}

