

\documentclass[aspectratio=169]{beamer}

\usepackage{amsmath,amssymb,amsfonts,amsthm}
%\usepackage[T1]{fontenc}
%\usepackage[utf8]{inputenc}
%\usepackage[english]{babel}
%\usepackage{lmodern}
\usepackage{comment}
\usepackage{xcolor}
\usepackage{graphicx}
\usepackage[linesnumbered,ruled,vlined,noend,algo2e]{algorithm2e}
\usepackage{tcolorbox}

\renewcommand{\epsilon}{\varepsilon}
\newcommand{\R}{\mathbb{R}}
\newcommand{\N}{\mathbb{N}}
\newcommand{\B}[2]{\mathcal{B}_{#1}\left(#2\right)}
\newcommand{\todo}[1]{{\color{red}#1}}

\newtcolorbox{myBox}{
tcbox width=auto limited,
capture=hbox,
colframe=red!100,
colback=red!10
}

\newcommand{\myBoxWrapper}[1]{
\begin{center}
    \begin{myBox}
        #1
    \end{myBox}
\end{center}
}

% Use Unipd as theme, with options:
% - pageofpages: define the separation symbol of the footer page of pages (e.g.: of, di, /, default: of)
% - logo: position another logo near the Unipd logo in the title page (e.g. department logo), passing the second logo path as option 
% Use the environment lastframe to add the endframe text
\usetheme[pageofpages=of]{Unipd}
\usefonttheme[onlymath]{serif}

\title{Density Based Clustering: a critical appraisal of DBSCAN}
\subtitle{}
\author[Simone Moretti]{Simone Moretti}

\date{May 2025 \\ {\small Mentor: Andrea Pietracaprina}}

% The next block of commands puts the table of contents at the beginning of each section and highlights the current section

\begin{comment}
\AtBeginSection[]
{
  \begin{frame}
    \frametitle{Table of Contents}
    \tableofcontents[currentsection]
  \end{frame}
}
\end{comment}


\begin{document}

% Make the title page
\frame{\titlepage}

\begin{frame}{Roadmap}

    \centering Let me take you on a trip through the history of DBSCAN!

    \vspace{0.3cm}

    \begin{figure}
        \centering
        \includegraphics[width=0.7\linewidth]{climbing_rock_mountain_climber_excursion_hiking_carega_pojesi-871204-2737140367.jpg}
    \end{figure}

\end{frame}

\begin{comment}
% Insert the general toc
\begin{frame}{Table of Contents}
    \tableofcontents    
\end{frame}
\end{comment}

\begin{frame}{Clustering}

    Group objects into \textbf{clusters}:

    \begin{itemize}
        \item Similar objects in the same cluster
        \item Different objects in different clusters
    \end{itemize}

    

    \vspace{0.3cm}
    
    \begin{figure}
        \centering
        \includegraphics[width=0.6\linewidth]{Clustering_Generale.png}
    \end{figure}

    
\end{frame}


\begin{frame}{Clustering on Metric Spaces}

    \begin{itemize}
        \item \textbf{Input}: a set $S$ of points in a \textbf{metric space}.
        \item \textbf{Output}: a clustering of the points based on their pairwise \textbf{distance} $d$ (computable in $O(1)$).
    \end{itemize}
    \begin{minipage}[c]{0.4\textwidth}
        \begin{figure}
            \centering
            \includegraphics[width=1\linewidth]{distance_between_two_points.png}
        \end{figure}
    \end{minipage}
    \begin{minipage}[c]{0.55\textwidth}
    \begin{tcolorbox}
        \begin{itemize}
            \item $d(x, y) \ge 0$
            \item $d(x,y) = 0 \iff x = y$
            \item $d(x,y) = d(y,x)$
            \item $d(x,y)\le d(x,z) + d(y,z)$
        \end{itemize}
    \end{tcolorbox}
    \end{minipage}


\end{frame}
\begin{comment}
    


\begin{frame}{Clustering Types}

Numerous types of clustering:

\begin{figure}
        \centering
        \includegraphics[width=0.5\linewidth]{clusteringtypes.jpg}
        \label{fig:enter-label}
    \end{figure}

        Focus of the presentation: \textbf{density-based} ones.\\

    \vspace{0.5cm}

\end{frame}
\end{comment}

\begin{frame}{Example: $k$-Means Clustering}

\begin{minipage}[c]{0.60\textwidth}
    Choose $k$ \textbf{centers} $\mathcal{C}=\{c_1,\dots,c_k\}$ in order to \textbf{minimize}:
   
    \[
        \sum_{x\in S} d(x,\mathcal{C})^2
    \]

    \vspace{0.3cm}
    \textbf{Clustering}: points assigned to their nearest center.
\end{minipage}
\hfill
\begin{minipage}[c]{0.35\textwidth}

    \begin{figure}
        \centering
        \includegraphics[width=0.75\linewidth]{k-means_unclustered.png}
    \end{figure}
    \begin{center}
        $\big\Downarrow$ $k$ = 3
    \end{center}
    \centering \includegraphics[width=0.75\linewidth]{k_means.png}

    
    
\end{minipage}

    
\end{frame}

\begin{comment}
\begin{frame}{Example: $k$-Means Clustering}
\begin{figure}
    \centering
    \includegraphics[width=0.9\linewidth]{kMeans.png}
    \caption{$k$-Means Clustering with $k=3$}
    \label{fig:enter-label}
\end{figure}
\end{frame}
\end{comment}

\begin{frame}{Example: $k$-Means Clustering}

    \begin{figure}
        \centering
        \includegraphics[width=0.6\linewidth]{Noisy_moons.png}
    \end{figure}

\end{frame}

\begin{frame}{Example: $k$-Means Clustering}

    \begin{figure}
        \centering
        \includegraphics[width=0.6\linewidth]{Noisy_moons_kMeans.png}
    \end{figure}
    \begin{center}
        \textit{Optimal $k$-Means clustering with $k=2$}
    \end{center}
\end{frame}

\begin{comment}
\begin{frame}{$k$-Means Clustering Problems}

    But wait, how do we cluster these sets? Clearly the previously cited algorithms perform badly.
    
\end{frame}


\end{comment}

\begin{frame}{Alternative Clustering}
    \centering Cluster together points that are \textbf{very close} to each other.

\vspace{0.3cm}

    \begin{figure}
        \centering
        \includegraphics[width=0.57\linewidth]{Noisy_moons.png}
    \end{figure}
   
\end{frame}

\begin{frame}{$\epsilon$-Neighbors}
   \centering Fix parameter $\epsilon\in\mathbb{R}_{>0}$. Points $x_1$ and $x_2$ are \textbf{$\mathbf{\epsilon}$-neighbors} if $d(x_1, x_2) \le \epsilon$.

\vspace{0.5cm}
    
    \begin{figure}
        \centering
        \includegraphics[width=0.5\linewidth]{neighboring.png}
    \end{figure}
    
\end{frame}

\begin{frame}{Alternative Clustering}

\begin{minipage}[c]{0.50\textwidth}
    Create undirected \textbf{graph} $G_\epsilon = (V,E_\epsilon)$.
    \begin{itemize}
        \item $V$ is the set $S$.
        \item An \textbf{edge} $\{x_1,x_2\}\in E_\epsilon$ if $x_1$ and $x_2$ are $\epsilon$-\textbf{neighbors}.
    \end{itemize}
\end{minipage}
\begin{minipage}[c]{0.05\textwidth}
    
\end{minipage}
\begin{minipage}[c]{0.45\textwidth}
    \begin{center}
        \includegraphics[width=0.95\linewidth]{Noisy_moons_graph.png}
        
         \textit{Graph $G_\epsilon$ for a suitable $\epsilon$}
        \end{center}
\end{minipage}
    
\end{frame}

\begin{frame}{Alternative Clustering}
    \centering Clusters of $G$ are its \textbf{connected components}!

    \vspace{0.3cm}

    \begin{figure}
        \centering
        \includegraphics[width=0.57\linewidth]{Connected_Components.png}
    \end{figure}
\end{frame}


\begin{emptyframe}
    Thank you!\\
    Any questions?
\end{emptyframe}

\begin{frame}{Alternative Clustering}

    \begin{figure}
        \centering
        \includegraphics[width=0.6\linewidth]{Noisy_connected.png}
        \label{fig:enter-label}
    \end{figure}
\end{frame}

\begin{frame}{Alternative Clustering}
    \begin{figure}
        \centering
        \includegraphics[width=0.6\linewidth]{noisy_component_one.png}
        \label{fig:enter-label}
    \end{figure}
\end{frame}

\begin{frame}{Density Based Clustering}

    Solution: \textbf{Density Based Clustering}.

    \vspace{0.3cm}

    \begin{minipage}[b]{0.20\textwidth}
        \begin{figure}
            \centering
            \includegraphics[width=1\linewidth]{martin_ester(1).jpg}
        \end{figure}
        \centering Martin Ester
    \end{minipage}
    \hfill
    \begin{minipage}[b]{0.20\textwidth}
        \begin{figure}
            \centering
            \includegraphics[width=1\linewidth]{hans_kriegel(1).jpg}
        \end{figure}
        \centering H.-P. Kriegel
    \end{minipage}
    \hfill
    \begin{minipage}[b]{0.20\textwidth}
        \begin{figure}
            \centering
            \includegraphics[width=1\linewidth]{joerg-sander(1).jpg}
        \end{figure}
        \centering Jörg Sander
    \end{minipage}
    \hfill
    \begin{minipage}[b]{0.20\textwidth}
        \begin{figure}
            \centering
            \includegraphics[width=1\linewidth]{Xiaowei_Xu(1).jpg}
        \end{figure}
        \centering Xiaowei Xu
    \end{minipage}

    \vspace{0.5cm}

    \textbf{DBSCAN} (1996)\cite{original_paper}: Density-Based Spatial Clustering of Applications with Noise.
\end{frame}


\begin{frame}{Density}
    
    A \textbf{ball} of radius $\epsilon$ centered around $x\in X$ is: \[\B{\epsilon}{x} = \{y\in X:d(x,y)\le\epsilon\}\]
    
    For parameter $MinPts\in\mathbb{Z}_{>0}$ and dataset $S$ a ball of radius $\epsilon$ is \textbf{dense} if it contains \textbf{at least} $MinPts$ points in $S$.

    \vspace{0.3cm}
    
  \centering
  \begin{minipage}[b]{0.45\textwidth}
    \begin{figure}
        \centering
        \includegraphics[width=0.45\linewidth]{denseball.png}
        \label{fig:enter-label}
    \end{figure}
    \begin{center}
        \textit{Dense ball}
    \end{center}
  \end{minipage}
  \hfill
  \begin{minipage}[b]{0.45\textwidth}
    \begin{figure}
        \centering
        \includegraphics[width=0.45\linewidth]{Nondenseball.png}
        \label{fig:enter-label}
    \end{figure}
    \begin{center}
        \textit{Non-dense ball}
    \end{center}
  \end{minipage}
  

    \raggedleft\footnotesize $MinPts=5$

    
\end{frame}
\begin{comment}
\begin{frame}{What is density?}
    We need to formalize the concept of density by some definitions:

    \begin{itemize}
        \item In a metric space $(X,d)$, a \textbf{ball} of radius $\epsilon$ centered around the point $x\in X$ is the set of points $\{y\in X:d(x,y)\le\epsilon\}$.
        \item Given $\epsilon\in\R$ and $MinPts\in\N$, we say that a ball of radius $\epsilon$ is \textbf{dense} if it contains at least $MinPts$ points.
        \item We also say that two points at distance $\le\epsilon$ are \textbf{neighbors}.
    \end{itemize}
    
\end{frame}
\end{comment}

\begin{frame}{Core, Border and Noise points}
    \noindent
    \begin{minipage}[t]{0.3\textwidth}
        
        \begin{figure}
            \centering
            \includegraphics[width=0.9\linewidth]{corepoint.png}
        \end{figure}
        
        A point $x$ is a \textbf{core point} if $\B{x}{\epsilon}$ is \textbf{dense}.
        
    
    \end{minipage}
    \hfill
    \begin{minipage}[t]{0.3\textwidth}
        \begin{figure}
            \centering
            \includegraphics[width=0.9\linewidth]{border.png}
        \end{figure}

        A point $x$ is a \textbf{border point} if $\B{x}{\epsilon}$ is \textbf{not dense} and it neighbors a core point.
    \end{minipage}
    \hfill
    \begin{minipage}[t]{0.3\textwidth}
        \begin{figure}
            \centering
            \includegraphics[width=1\linewidth]{noisepoint.png}
        \end{figure}
        
        A point $x$ is a \textbf{noise point} if it is neither a core point or a border one.
        
    \end{minipage}

\vspace{1.1cm}

    
    \raggedleft\footnotesize $MinPts=5$
    

\end{frame}

\begin{comment}
    \begin{frame}{DBSCAN Clustering}

   DBSCAN clusters are based on the connected components of graph $G$ made of \textbf{core points}.

    \vspace{0.5cm}

    \begin{figure}
        \centering
        \includegraphics[width=0.7\linewidth]{corepointsconnectedcomp.png}
        \label{fig:enter-label}
    \end{figure}
    
\end{frame}

\begin{frame}{DBSCAN Clustering}
    Clusters should also contain \textbf{border points} adjacent to their core points.
    \vspace{0.5cm}
    \begin{figure}
        \centering
        \includegraphics[width=0.7\linewidth]{dbscan_add_border_points.png}
    \end{figure}
    
\end{frame}
\end{comment}

\begin{frame}{DBSCAN Clustering}
    \begin{minipage}[c]{0.55\textwidth}
        \begin{figure}
            \centering
            \includegraphics[width=0.9\linewidth]{Noisy_moons_graph_2.png}
        \end{figure}
    \end{minipage}
    \hfill
    \begin{minipage}[c]{0.44\textwidth}
        1. Find the \textbf{graph} $G$ of $\epsilon$ neighbors.
    \end{minipage}
\end{frame}

\begin{frame}{DBSCAN Clustering}
    \begin{minipage}[c]{0.55\textwidth}
        \begin{figure}
            \centering
            \includegraphics[width=0.9\linewidth]{allcorepoints.png}
        \end{figure}
    \end{minipage}
    \hfill
    \begin{minipage}[c]{0.44\textwidth}
        2. Find all \textbf{core points}.
    \end{minipage}
\end{frame}

\begin{frame}{DBSCAN Clustering}
    \begin{minipage}[c]{0.55\textwidth}
        \begin{figure}
            \centering
            \includegraphics[width=0.9\linewidth]{corepointsconnectedcomp.png}
        \end{figure}
    \end{minipage}
    \hfill
    \begin{minipage}[c]{0.44\textwidth}
        3. Find \textbf{connected components} made of core points.
    \end{minipage}
\end{frame}

\begin{frame}{DBSCAN Clustering}
    \begin{minipage}[c]{0.55\textwidth}
        \begin{figure}
            \centering
            \includegraphics[width=0.9\linewidth]{dbscan_add_border_points.png}
        \end{figure}
    \end{minipage}
    \hfill
    \begin{minipage}[c]{0.44\textwidth}
        4. Add \textbf{border points} and return the \textbf{clusters} found.
    \end{minipage}
\end{frame}

\begin{frame}{Noise points}

    \begin{minipage}[c]{0.55\textwidth}
        \begin{figure}
        \centering
        \includegraphics[width=0.9\linewidth]{DBSCAN_noisy_points.png}
    \end{figure}
    \end{minipage}
    \hfill
    \begin{minipage}[c]{0.44\textwidth}
        \textbf{Noise points} will be left unclustered.
    \end{minipage}    
    
\end{frame}

\begin{comment}
\input{exact}
\end{comment}

\begin{frame}{Border Points}
   \centering Border points are part of \textbf{every cluster} adjacent to them.
    \begin{figure}
        \centering
        \includegraphics[width=0.7\linewidth]{Exact/border_problem.png}
    \end{figure}

   \centering This ensures that the clustering is \textbf{unique} (for fixed $\epsilon$ and  $MinPts$).
\end{frame}


\begin{frame}{Region Query}

    Find out neighbors of point $x$: \textbf{regionQuery}($x$,$\epsilon$).\\

    \vspace{0.5cm}

    Trivial implementation:

    \vspace{0.5cm}

    \begin{algorithm2e}[H] 
\SetAlgoLined
    $neighbors\leftarrow\emptyset$
    
    \For{$q\in S$}{
        \If{$d(x,q)\le\epsilon$}{
            $neighbors\leftarrow neighbors\cup\{q\}$
        }
    }
    \Return{$neighbors$}
    
\caption{$\textsc{regionQuery}(x, \epsilon)$} %\label{alg:regionQuery}
\end{algorithm2e}

\end{frame}

\begin{frame}{DBSCAN Clustering}
    \begin{minipage}[c]{0.55\textwidth}
        \begin{figure}
            \centering
            \includegraphics[width=0.9\linewidth]{Noisy_moons_graph_2.png}
        \end{figure}
    \end{minipage}
    \hfill
    \begin{minipage}[c]{0.44\textwidth}
        {\color{gray}1. Find the \textbf{graph} $G$ of $\epsilon$ neighbors.}\\
        We compute regionQuery on each point to find all edges of $G$.
    \end{minipage}
\end{frame}

\begin{frame}{DBSCAN Clustering}
    \begin{minipage}[c]{0.55\textwidth}
        \begin{figure}
            \centering
            \includegraphics[width=0.9\linewidth]{allcorepoints.png}
        \end{figure}
    \end{minipage}
    \hfill
    \begin{minipage}[c]{0.44\textwidth}
        {\color{gray}2. Find all \textbf{core points}.}\\
        A point $x$ is a core point iff $|regionQuery(x,\epsilon)|\ge MinPts$.
    \end{minipage}
\end{frame}

\begin{frame}{DBSCAN Clustering}
    \begin{minipage}[c]{0.55\textwidth}
        \begin{figure}
            \centering
            \includegraphics[width=0.9\linewidth]{corepointsconnectedcomp.png}
        \end{figure}
    \end{minipage}
    \hfill
    \begin{minipage}[c]{0.44\textwidth}
        {\color{gray}3. Find \textbf{connected components} made of core points.}\\
        Finding connected components is linear on the size of the graph using BFS.
    \end{minipage}
\end{frame}

\begin{frame}{DBSCAN Clustering}
    \begin{minipage}[c]{0.55\textwidth}
        \begin{figure}
            \centering
            \includegraphics[width=0.9\linewidth]{dbscan_add_border_points.png}
        \end{figure}
    \end{minipage}
    \hfill
    \begin{minipage}[c]{0.44\textwidth}
        {\color{gray}4. Add \textbf{border points} and return the \textbf{clusters} found.}\\
        Find border points by checking if unclustered points neighbor a core point.
    \end{minipage}
\end{frame}

\begin{comment}
\begin{frame}{DBSCAN Algorithm using regionQuery}
    \begin{minipage}[c]{0.44\textwidth}
        \begin{figure}
            \centering
            \includegraphics[width=0.9\linewidth]{Noisy_moons_graph_2.png}
        \end{figure}
    \end{minipage}
    \hfill
    \begin{minipage}[c]{0.5\textwidth}
        {\color{gray}1. Find the \textbf{graph} $G$ of $\epsilon$ neighbors.}\\
        We compute regionQuery on each point to find all edges of $G$.
    \end{minipage}

    \begin{minipage}[c]{0.44\textwidth}
        \begin{figure}
            \centering
            \includegraphics[width=0.9\linewidth]{allcorepoints.png}
        \end{figure}
    \end{minipage}
    \hfill
    \begin{minipage}[c]{0.5\textwidth}
        {\color{gray}2. Find all \textbf{core points}.}\\
        A point $x$ is a core point iff $|regionQuery(x,\epsilon)|\ge MinPts$.
    \end{minipage}


    
\end{frame}

\begin{frame}{DBSCAN Algorithm using regionQuery}

    \begin{minipage}[c]{0.44\textwidth}
        \begin{figure}
            \centering
            \includegraphics[width=0.9\linewidth]{corepointsconnectedcomp.png}
        \end{figure}
    \end{minipage}
    \hfill
    \begin{minipage}[c]{0.5\textwidth}
        {\color{gray}3. Find \textbf{connected components} made of core points.}\\
        Finding connected components is linear on the size of the graph using BFS.
    \end{minipage}


    \begin{minipage}[c]{0.44\textwidth}
        \begin{figure}
            \centering
            \includegraphics[width=0.9\linewidth]{dbscan_final.png}
        \end{figure}
    \end{minipage}
    \hfill
    \begin{minipage}[c]{0.5\textwidth}
        {\color{gray}4. Add \textbf{border points} and return the \textbf{clusters} found.}\\
        Find border points by checking if unclustered points neighbor a core point.
    \end{minipage}

    
\end{frame}
\end{comment}

\begin{frame}{Running Time}
    Running time bounded by the calls of the regionQuery function:
    \begin{itemize}
        \item Every point calls regionQuery a costant number of times.
        \item The trivial implementation runs in $O(n)$ ($n:=|S|$).
    \end{itemize}

    \vspace{0.5cm}

    \textbf{Total running time}: \[\Large{O(n^2)}\]
\end{frame}

\begin{frame}{Running Time}
    Original authors claimed an implementation of regionQuery using \textbf{R*-trees} that runs in $O(\log n)$ on average. \textbf{Total running time}:
    \[
        O(n\log n)
    \]

    \vspace{0.3cm}

    \begin{figure}
        \centering
        \includegraphics[width=0.67\linewidth]{R-tree.png}
    \end{figure}
    \begin{center}
        \textit{R*-tree}
    \end{center}

\end{frame}

\begin{frame}{Running Time}
    Claim \textbf{not rigorous}, from later studies we now know that:
    \begin{itemize}
        \item On $\mathbb{E}^d$ ($d\ge3)$ DBSCAN is as \textbf{hard as USEC} (likely $\Omega(n^{4/3})$).
        \item On $\mathbb{E}^d$ ($d\ge5)$ DBSCAN is as \textbf{hard as Hopcroft} (likely $\Omega(n^{4/3})$).
        \item On non Euclidean spaces things might be even more complicated.
    \end{itemize}

\vspace{0.5cm}

    \begin{minipage}[c]{0.45\textwidth}
        \begin{figure}
            \centering
            \includegraphics[width=0.65\linewidth]{usec.png}
        \end{figure}
        \begin{center}
            \footnotesize\textit{USEC}
        \end{center}
    \end{minipage}
    \hfill
    \begin{minipage}[c]{0.45\textwidth}
        \begin{figure}
            \centering
            \includegraphics[width=0.65\linewidth]{Hopcroft.png}
        \end{figure}
        \begin{center}
            \footnotesize\textit{Hopcroft}
        \end{center}
    \end{minipage}
    
    
\end{frame}

\begin{comment}
\begin{frame}{DBSCAN pros and cons}
    Advantages of the DBSCAN algorithm:
    \begin{itemize}
        \item Finds \textbf{various shaped clusters} and isn't limited to convex ones.
        \item It is an \textbf{exact} algorithm.
        \item It doesn't assume the number of clusters in advance.
        \item It is extremely \textbf{fast} as it runs in $O(n\log n)$
    \end{itemize}

    Disadvantages of the DBSCAN algorithm:
    \begin{itemize}
        \item We still need to tune parameters $\epsilon$ and $MinPts$.
        \item It doesn't actually run in $O(n\log n)$!!
    \end{itemize}
\end{frame}

\begin{frame}{Too good to be true}

    The original claim was a bit optimistic.

    \vspace{0.5cm}
    
    When all points are at distance $\le\epsilon$, regionQuery returns the entire set. Total running time: $\Theta(n^2)$.

    \begin{figure}
        \centering
        \includegraphics[width=0.35\linewidth]{close_points.png}
    \end{figure}

    

    
\end{frame}

\begin{frame}{Hardness of exact DBSCAN}
    Junhao Gan and Yufei Tao (2013)\cite{follow_up}:
    \begin{itemize}
        \item On the Euclidean Plane the complexity can be improved to $O(n\log\log n)$.
        \item For $d\geq3$, the problem is as hard as USEC, which has a believed complexity of $\Omega(n^{4/3})$.
        \item For $d\geq5$, the problem is is as hard as Hopcroft, which has a believed complexity of $\Omega(n^{4/3})$.
    \end{itemize}
\end{frame}

\begin{frame}{Hardness of exact DBSCAN}

    All these results cover the Euclidean space, for a general metric space we could assume that the algorithm is $O(n^2)$.

    The problem is much harder than it seemed at first.
    
\end{frame}

\end{comment}
\begin{comment}
\begin{frame}{Recap}

    open problems: spazi ad alte dimensioni + spazi metrici generali
\end{frame}
\end{comment}

\begin{frame}{Gan \& Tao}
    Previous results discovered by:
    \vspace{0.3cm}
    
    \begin{minipage}[b]{0.45\textwidth}
    \centering
        \begin{figure}
            \centering
            \includegraphics[width=0.45\linewidth]{yufei-tao.jpg}
        \end{figure}
        Yufei Tao
    \end{minipage}
    \hfill
    \begin{minipage}[b]{0.45\textwidth}
    \centering
        \begin{figure}
            \centering
            \includegraphics[width=0.45\linewidth]{Junhao-gan.jpg}
        \end{figure}
        Junhao Gan
    \end{minipage}

    \vspace{0.3cm}
    
    They also found an \textbf{approximate solution} in the $d$-dimensional Euclidean space which runs in $O(n)$ \cite{follow_up}.
\end{frame}

\begin{comment}
\begin{frame}{Approximate Solution}
    In computer science, when a problem is too hard, we start looking for \textbf{approximate solutions}. 

    In our case it is apparently difficult to define an approximate solution, since DBSCAN is not an optimization problem.

    We will not approximate on the value of the solution, instead we will try to find a clustering that is in some sense approximately equal to the clustering we want to find.
\end{frame}
\end{comment}

\begin{frame}{Reminder}
   \centering Fix parameter $\epsilon\in\mathbb{R}_{>0}$. Points $x_1$ and $x_2$ are \textbf{neighbors} if $d(x_1, x_2) \le \epsilon$.

    \vspace{0.5cm}
    
    \begin{figure}
        \centering
        \includegraphics[width=0.5\linewidth]{neighboring.png}
    \end{figure}
    
\end{frame}

\begin{frame}{$\rho$-Approximation}
    Fix parameter $\rho\in\mathbb{R}_{>0}$. \\
    \textbf{$(\rho,\epsilon)$-Neighboring}: points at distance $\epsilon\le d\le (1+\rho)\epsilon$ might be classified as neighbors.

    \begin{figure}
        \centering
        \includegraphics[width=0.5\linewidth]{rho-approx.png}
    \end{figure}

    
\end{frame}


\begin{frame}{Approximate our algorithm}
    Can we use the concept of approximate neighbor to improve the previous DBSCAN algorithm?\\

    \vspace{0.5cm}

    \begin{minipage}{0.32 \textwidth}
        \begin{figure}
            \centering
            \includegraphics[width=\linewidth]{Noisy_moons_graph_2.png}
        \end{figure}
    \end{minipage}
    \hfill
    \begin{minipage}{0.32 \textwidth}
        \begin{figure}
            \centering
            \includegraphics[width=\linewidth]{allcorepoints.png}
        \end{figure}
    \end{minipage}
    \hfill
    \begin{minipage}{0.32 \textwidth}
        \begin{figure}
            \centering
            \includegraphics[width=\linewidth]{dbscan_final.png}
        \end{figure}
    \end{minipage}
    
\end{frame}



\begin{frame}{Compute The Graph $G$}

    \centering We are limited by the \textbf{size} of graph $G$!

    \vspace{0.3cm}
    
    \begin{figure}
        \centering
        \includegraphics[width=0.5\linewidth]{Dense_graph.png}
    \end{figure}
    \begin{center}
        \textit{A complete graph emerges with $\epsilon$ large enough}
    \end{center}
    
\end{frame}

\begin{frame}{Grid Subdivision}

    Divide the Euclidean space in \textbf{grid cells} of side length $\frac{\epsilon}{\sqrt{d}}$.
    
\vspace{0.3cm}

    \begin{minipage}[c]{0.45\textwidth}
        \begin{figure}
            \centering
            \includegraphics[width=0.9\linewidth]{Grid.png}
        \end{figure}
    \end{minipage}
    \begin{minipage}[c]{0.50\textwidth}
        \begin{itemize}
       \item Length of the major diagonal of a cell is $\epsilon$. \\
    
        \item Two points in the same cell are surely neighbors.
        \end{itemize}
    \end{minipage}

    
\end{frame}

\begin{frame}{Graph on Cells}

    Instead of creating a graph on the points we can create a graph on the cells.

\vspace{0.3cm}

    \begin{minipage}[c]{0.45\textwidth}
        \begin{figure}
        \centering
        \includegraphics[width=0.9\linewidth]{graph_on_grid_cells.png}
    \end{figure}
    \end{minipage}
    \begin{minipage}[c]{0.50\textwidth}
        Two cells are neighbors if they contain two neighboring \textbf{core} points.
    \end{minipage}

    

    
    
\end{frame}

\begin{frame}{Graph on Cells}

    Each cell has side length $l =\frac{\epsilon}{\sqrt{d}}$, so it has $O\left(\left(\frac{\epsilon}{l}\right)^d\right)= O\left(d^{d/2}\right)$ neighboring cells.

    \vspace{0.3cm}

    \begin{figure}
        \centering
        \includegraphics[width=0.25\linewidth]{neighboring_cells.png}
    \end{figure}

    \vspace{0.3cm}

    Total size of graph: $O\left(n\cdot d^{d/2}\right)=O(n)$
    
\end{frame}

\begin{frame}{Find Core Points}
    Function ApproxRangeCount$(x,\epsilon,\rho)$:
    
\vspace{0.3cm}
    \begin{minipage}{0.55\textwidth}
    \begin{itemize}
        \item Returns \textbf{approximate number of neighbors} of point $x$.
        \item Can be implemented in $O\left(\left(\frac{1}{\rho}\right)^d\right)=O(1)$.
    \end{itemize}
        
    \end{minipage}
    \hfill
    \begin{minipage}{0.4\textwidth}
        \begin{figure}
            \centering
            \includegraphics[width=0.8\linewidth]{Approximaterangecounting.png}
        \end{figure}
    \end{minipage}
    
    
\vspace{0.3cm}

\end{frame}

\begin{frame}{Find Core Points}
    Function ApproxRangeCount$(x,\epsilon,\rho)$:
    
\vspace{0.3cm}
    \begin{minipage}{0.55\textwidth}
    \begin{itemize}
        \item Returns \textbf{approximate number of neighbors} of point $x$.
        \item Can be implemented in $O\left(\left(\frac{1}{\rho}\right)^d\right)=O(1)$.
    \end{itemize}
        
    \end{minipage}
    \hfill
    \begin{minipage}{0.4\textwidth}
        \begin{figure}
            \centering
            \includegraphics[width=0.8\linewidth]{Approximaterangecounting.png}
        \end{figure}
    \end{minipage}
    
    
\vspace{0.3cm}

        \centering $x$ is a \textbf{core point} if ApproxRangeCount$(x,\epsilon,\rho)\ge MinPts$.
\end{frame}

\begin{frame}{Core Cells}
    A cell is a \textbf{core cell} if it contains a core point.
    
    \vspace{0.3cm}
    
    \begin{figure}
        \centering
        \includegraphics[width=0.3\linewidth]{Core_cells.png}
    \end{figure}

    As before, we will find connected components consisting only of core cells.
\end{frame}

\begin{frame}{Find Neighboring Core Cells}
    Check if core cells $c_1$ and $c_2$ are neighboring:
    \begin{itemize}
        \item \textbf{Slow Method}: check the distances of all pairwise core points.
    \end{itemize}

    \begin{figure}
        \centering
        \includegraphics[width=0.5\linewidth]{Slow_neighboring_check_core_cells.png}
    \end{figure}
    
    Running time might be $\Omega(n^2)$.
    
\end{frame}

\begin{frame}{Find Neighboring Core Cells}
    Check if core cells $c_1$ and $c_2$ are neighboring:
    \begin{itemize}
        \item \textbf{Fast Method}: $\forall x\in c_1$ core point run modified ApproxRangeCount$(x,\epsilon,\rho, c_2)$ that only counts core points in cell $c_2$ in $O(1)$.
        \item If ApproxRangeCount$(x,\epsilon,\rho, c_2)\ge 1$ then $x$ has a neighbor in $c_2$.
    \end{itemize}

    \begin{figure}
        \centering
        \includegraphics[width=0.3\linewidth]{Fast_neighboring_check_core_cells.png}
    \end{figure}

    Running time is $O(n)$.
    
\end{frame}

\begin{frame}{Approximate DBSCAN}
    \begin{enumerate}
        \item Subdivide the space in a grid.
        \item Find all core points and core cells.
        \item For each core cell find its neighboring cells.
        \item Find connected components of core cells.
        \item Cluster points based on the connected component of their core cell.
        \item Add border points.
        \item Return the found clusters.
    \end{enumerate}
\end{frame}


\begin{frame}{Sandwich Theorem}
    Guarantee on the \textbf{quality} of an approximate solution:\\
    $\forall C\in$ ApproxDBSCAN$(\epsilon,\rho)$ $\exists C'\in$ DBSCAN$(\epsilon)$ and $C''\in$ DBSCAN$((1+\rho)\epsilon)$ s.t.: $C'\subseteq C\subseteq C''$
        \vspace{0.5cm}
    
    \begin{minipage}{0.3\textwidth}
        \begin{figure}
            \centering
            \includegraphics[width=1\linewidth]{Sandwich_clusters1.png}
        \end{figure}
        \begin{center}
            \small \textit{Exact solution for $\epsilon$}
        \end{center}
    \end{minipage}
    \begin{minipage}{0.3\textwidth}
        \begin{figure}
            \centering
            \includegraphics[width=1\linewidth]{Sandwich_clusters2.png}
        \end{figure}
        \begin{center}
            \small \textit{Approximate solution}
        \end{center}
    \end{minipage}
    \begin{minipage}{0.3\textwidth}
        \begin{figure}
            \centering
            \includegraphics[width=1\linewidth]{Sandwich_clusters3.png}
        \end{figure}

        \begin{center}
            \small \textit{Exact solution for $(1+\rho)\epsilon$}
        \end{center}
    \end{minipage}
    
\end{frame}

\begin{frame}{Experiments}
    The approximate algorithm is way \textbf{faster} than the original one.

    \vspace{0.5cm}

    \begin{figure}
    \centering
    \includegraphics[width=0.57\linewidth]{approx_dbscan_running_time.png}
\end{figure}
\begin{center}
\textit{Running time of approximate DBSCAN from Gan \& Tao}
\end{center}
\end{frame}

\begin{frame}{Experiments}

    On real-life datasets a value of $\rho=10^{-2}$ almost always results in an approximated solution equal to the exact one.

    \vspace{0.5cm}

    \begin{figure}
        \centering
        \includegraphics[width=1\linewidth]{error_dbscan.png}
    \end{figure}
    
\end{frame}

\begin{frame}{Further Research}

    In 2024 Mo and Shong investigated DBSCAN on dataset with low \textbf{doubling dimension} \cite{doubling_dimension}.

    \vspace{0.5cm}

    \centering \fbox{\includegraphics[width=0.85\linewidth]{low_doubling_dimension.png}}
\end{frame}

\begin{frame}{Test of Time Award}
    In 2014 the original DBSCAN paper wins the \textbf{SIGKDD Test of Time Award}.

\vspace{0.5cm}
    
    \begin{figure}
        \centering
        \includegraphics[width=0.5\linewidth]{test_of_time_award.jpg}
    \end{figure}
    
\end{frame}

\begin{frame}{Applications}
    DBSCAN has many \textbf{real world applications}:

    \vspace{0.3cm}

    \begin{minipage}{0.3\textwidth}
        \centering
        \textbf{Bioinformatics}
        \begin{figure}
            \centering
            \includegraphics[width=0.9\linewidth]{shutterstock_430479463-e1563327818340-376507331.jpg}
        \end{figure}
        \begin{itemize}
        \item Protein family identification
        \item Metagenomic sequence analysis
    \end{itemize}
    \end{minipage}
    \begin{minipage}{0.3\textwidth}
        \centering
        \textbf{Cybersecurity}
        \begin{figure}
            \centering
            \includegraphics[width=0.9\linewidth]{gettyimages-1370834192-2668288701.jpg}
        \end{figure}
        \begin{itemize}
        \item Network intrusion detection
        \item Fraud pattern discovery
    \end{itemize}
    \end{minipage}
    \begin{minipage}{0.3\textwidth}
        \centering
        \textbf{Computer Vision}
        \begin{figure}
            \centering
            \includegraphics[width=0.9\linewidth]{computer-vision-3713535937.jpg}
        \end{figure}
        \begin{itemize}
        \item Image segmentation
        \item LiDAR point cloud processing
        \item Anomaly detection
    \end{itemize}
    \end{minipage}
\end{frame}

\begin{frame}{LiDAR Point Cloud Processing}


\begin{minipage}[c]{0.8\textwidth}
    \textbf{Alessandro Dario} and his team \textbf{RaceUp} use DBSCAN for detecting cones in the path of their \textbf{Formula Student} car!
\end{minipage}
\begin{minipage}[c]{0.17\textwidth}
    \begin{figure}
        \centering
        \includegraphics[width=1\linewidth]{Alessandro_Dario.jpg}
    \end{figure}
\end{minipage}

\vspace{0.3cm}

\begin{minipage}[c]{0.4\textwidth}
    \begin{figure}
        \centering
        \includegraphics[width=1\linewidth]{formula_student_car.png}
    \end{figure}
\end{minipage}
\hfill 
\begin{minipage}[c]{0.55\textwidth}

\begin{figure}
    \centering
    \includegraphics[width=\linewidth]{formula_student.png}
\end{figure}
\end{minipage}
\end{frame}


\begin{comment}
\begin{frame}{The Future of DBSCAN}
    DBSCAN still remains an \textbf{active research field} to this day and only awaits for passionate researchers to develop further.

    \begin{figure}
        \centering
        \includegraphics[width=0.65\linewidth]{38-381783_want-you-for-us-army-transparent-hd-png-4253689009.png}
    \end{figure}
\end{frame}
\end{comment}

\begin{comment}
\begin{frame}{DBSCAN Recap}

\begin{itemize}
    \item In 2014 DBSCAN wins the \textbf{SIGKDD Test of Time Award}.
    \item Still used in many real world application in its original form.
    \item Has great runing time improvement with approximation techniques.
    \item Still an active research field, especially in non-Euclidean Spaces.


    
\end{itemize}
    
\end{frame}

\begin{frame}{Real World Applications}
    \todo{OLD}
    
    The original paper won the 2014 SIGKDD Test of Time Award and it's still a widely used algorithm in many applications in:
    \begin{columns}
    \column{0.5\textwidth}

    \textbf{Computer Vision}
    \begin{itemize}
        \item Image segmentation
        \item LiDAR point cloud processing
        \item Anomaly detection in MRI scans
    \end{itemize}

    \column{0.5\textwidth}
    \textbf{Cybersecurity}
    \begin{itemize}
        \item Network intrusion detection
        \item Fraud pattern discovery
    \end{itemize}
    
    \textbf{Bioinformatics}
    \begin{itemize}
        \item Protein family identification
        \item Metagenomic sequence analysis
    \end{itemize}
\end{columns}
\end{frame}


\begin{frame}{Formula Student}
    \cite{follow_follow_up}
\end{frame}

\end{comment}

\begin{frame}{Bibliography}
    \bibliographystyle{abbrv}
    \bibliography{biblio}
\end{frame}

\begin{emptyframe}
    Thank you!\\
    Any questions?
\end{emptyframe}


\begin{frame}{Quad Trees}
    \begin{figure}
        \centering
        \includegraphics[width=0.9\linewidth]{quad_tree.png}
    \end{figure}
\end{frame}

\begin{frame}{Doubling Dimension}
    Let $\Lambda$ be the smallest positive integer s.t. $\forall x\in S$ and $\forall \epsilon>0$, $S\cap\B{2\epsilon}{x}$ can always be covered by the union of at most $\Lambda$ balls with radius $\epsilon$.
    Then the \textbf{Doubling Dimension} of $S$ is:
    \[D = \lceil\log_2\Lambda\rceil\]

    \begin{figure}
        \centering
        \includegraphics[width=0.35\linewidth]{doubling_dimension.png}
    \end{figure}
\end{frame}

\end{document}

